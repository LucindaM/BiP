%	Use the standard preamble for Beamer slides of all 
%		statsTeachR modules
%		(\input, not \include, as \include can't access 
%		things in higher-level directories since it needs 
%		write permission there, which it doesn't have; and 
%		in some settings the preamble may be in a higher-level
%		directory than the source file.)
%	This path assumes the preamble is in the parent directory,
%		modify this if that changes.
%	************************************************
%	**	LaTeX preamble to be used with all 
%	**	statsTeachR labs/handouts.
%
%	Author: Eric A. Cohen
%	Last modified: 22 August 2013
%	************************************************

\documentclass[table]{beamer}

%	Set theme (a nice plain one)
\usetheme{Malmoe}

%	Use named colors, set main color of theme
%		to match Web site color:
\definecolor{MainColor}{RGB}{10, 74, 109}
\colorlet{MainColorMedium}{MainColor!50}
\colorlet{MainColorLight}{MainColor!20}
\usecolortheme[named=MainColor]{structure} 

%	For tables
%[dvipsnames] [table]
\usepackage{xcolor}
\usepackage{tabu}	% Even fancier than tabulary
\usepackage{multirow}

%	Just for the degree symbol
\usepackage{textcomp}

%	Get rid of footline (page, author, etc. on each slide)
\setbeamertemplate{footline}{}
%	Get rid of navigation buttons
\setbeamertemplate{navigation symbols}{}

%	Make footnotes not ugly
\usepackage{hanging}
\setbeamertemplate{footnote}{\raggedright\hangpara{1em}{1}\makebox[1em][l]{\insertfootnotemark}\footnotesize\insertfootnotetext\par}

%	Text style for code snippets inline in text:
\newcommand{\codeInline}[1]{\texttt{#1}}

%	Text style for emphasis stronger than \emph:
%		(Note, this doesn't toggle the way \emph does.
%			(Note, this can be done, didn't seem worth the trouble.))
\newcommand{\strong}[1]{{\bfseries{#1}}}


%	******	Define title page	**********************
\setbeamertemplate{title page}{
	{\color{MainColor}
	% There must be a better way than this -vspace at
	%	 the top and bottom of the page to reduce the 
	%	 bottom margin, but I can't find one that works.
	\vspace{-6em}

	% Go to a lot of trouble to get the title in a
	%	nice box, since customizing a beamer block
	%	does not entirely work here (I don't know why)
	\newlength{\titleBoxWidth}
	\setlength{\titleBoxWidth}{\textwidth}
	\addtolength{\titleBoxWidth}{-2.0em}
	\setlength{\fboxsep}{1.0em}
	\setlength{\fboxrule}{0pt}
	\fcolorbox{MainColor!25}{MainColor!25}{
		\parbox{\titleBoxWidth}{
			\raggedright
			\LARGE\textbf{\inserttitle}
		}	% end parbox
	}	% end fcolorbox

	\vfill
	\small{Author: \insertauthor}
	\vspace{\baselineskip}

	\small{\Course}

	\small{\Instructor}
	\vspace{\baselineskip}

	\small{\emph{This material is part of the \strong{statsTeachR} project}}

	\vspace{0.33\baselineskip}\scriptsize{\emph{\LicenseText}}





		\vspace{-15em}

	}	% end color
	\clearpage
}	% end define title page




%	The following variables are assumed by the standard preamble:
%	Global variable containing module name:
\title{Principles of Reproducible Research with R}
%	Global variable containing module shortname:
%		(Currently unused, may be used in future.)
\newcommand{\ModuleShortname}{shortName}
%	Global variable containing author name:
\author{Andrea Foulkes, Gregory Matthews, Nicholas Reich}
%	Global variable containing text of license terms:
\newcommand{\LicenseText}{Made available under the Creative Commons Attribution-ShareAlike 3.0 Unported License: http://creativecommons.org/licenses/by-sa/3.0/deed.en\textunderscore US }
%	Instructor: optional, can leave blank.
%		Recommended format: {Instructor: Jane Doe}
\newcommand{\Instructor}{}
%	Course: optional, can leave blank.
%		Recommended format: {Course: Biostatistics 101}
\newcommand{\Course}{Biostatistics in Practice: Research Training in High-Performance Computing with R}



%	******	Document body begins here	**********************

\begin{document}

%	Title page
\begin{frame}[plain]
	\titlepage
\end{frame}

%	******	Everything through the above line must be placed at
%		the top of any TeX file using the statsTeachR standard
%		beamer preamble. 


\begin{frame}{Introduction and welcome...}

\end{frame}

\begin{frame}{Tools for reproducible research}

``dynamic documents''

\end{frame}


\begin{frame}{Version control and reproducibility}

\end{frame}

\begin{frame}{GitHub increases productivity and transparency}

\end{frame}


%%%%%%%%%%%%%%%%%%%%%%%%%%%%%%%%%%%%%%%%%%%%%%%%%%%%%%%
%%% TEMPLATES FOR STATSTEACHR FORMATTING BELOW HERE  %%
%%%%%%%%%%%%%%%%%%%%%%%%%%%%%%%%%%%%%%%%%%%%%%%%%%%%%%%

% \begin{frame}{Overview}
% 
% 	\begin{block}{Diagnostic Test}
% 		Tool for deciding on something you want to know --- the true status of some outcome --- based on something you can measure.
% 
% 		\begin{itemize}
% 
% 			\item{Deciding whether someone has tuberculosis (the outcome) based on the presence or size of a raised swelling (the test) following tuberculin injection.}
% 
% 			\item{Deciding whether an infant has phenylketonuria (the outcome) based on the blood level of phenylalanine (the test).}
% 
% 			\item{Deciding whether an infant has phenylketonuria (the outcome) based on the odor of the urine (same outcome of interest, different test).}
% 		\end{itemize}
% 
% 	\end{block}
% 
% \end{frame}
% 
% 
% 
% \begin{frame}{The Gold Standard}
% 
% In evaluating a test, must compare results to some ``truth''.
% 
% This is provided by the best possible test --- the \emph{gold standard} or \emph{criterion standard} test.
% 
% \medskip
% 
% 	\begin{block}{Why Not Use the Gold Standard?}
% 
% 		 May be expensive, invasive, or impractical.
% 
% 		\begin{itemize}
% 
% 			\item{Gold-standard test for pancreatic cancer involves biopsy of the pancreas;}
% 			\item{Gold-standard test for Alzheimer's disease is post-mortem brain autopsy.}
% 
% 		\end{itemize}
% 
% 	\end{block}
% 
% \end{frame}
% 
% 
% 
% \begin{frame}{Sensitivity and Specificity}
% 
% \taburulecolor{MainColorMedium}
% \tabulinesep=1.5ex
% \begin{tabu} to 1.0\textwidth[r]{X[,L,]X[,L,]}
% 	\rowcolor{MainColorLight}\strong{Sensitivity}                                                                    	&
% 	\strong{Specificity}
% \\
% 	How good is this test at \textit{detecting cases} of this condition?
% 	&
% 	How good is this test at giving a \textit{clean bill of health} to those \textit{without} the condition?
% \\\hline
% 	What proportion, of those with the condition, test positive for the condition?
% 	&
% 	What proportion, of those without the condition, test negative for the condition?
% \\\hline
% 	\vspace{-3ex}\begin{displaymath}\frac{true\ positives}{positives\ in\ population}\end{displaymath}\vspace{-3ex}
% 	&
% 	\vspace{-3ex}\begin{displaymath}\frac{true\ negatives}{negatives\ in\ population}\end{displaymath}\vspace{-3ex}
% \\\hline
% 	\begin{math}
% 		Pr(T^{+} \mid D^{+})
% 	\end{math}
% 	\newline
% 	$T^{+}$ --- positive test;\newline
% 	$D^{+}$ --- disease
% 	&
% 	\begin{math}
% 		Pr(T^{-} \mid D^{-})
% 	\end{math}
% 	\newline
% 	$T^{-}$ --- negative test;\newline
% 	$D^{+}$ --- no disease
% \\
% \end{tabu}
% 
% \end{frame}


\end{document}