%	Use the standard preamble for Beamer slides of all 
%		statsTeachR modules
%		(\input, not \include, as \include can't access 
%		things in higher-level directories since it needs 
%		write permission there, which it doesn't have; and 
%		in some settings the preamble may be in a higher-level
%		directory than the source file.)
%	This path assumes the preamble is in the parent directory,
%		modify this if that changes.
%	************************************************
%	**	LaTeX preamble to be used with all 
%	**	statsTeachR labs/handouts.
%
%	Author: Eric A. Cohen
%	Last modified: 22 August 2013
%	************************************************

\documentclass[table]{beamer}

%	Set theme (a nice plain one)
\usetheme{Malmoe}

%	Use named colors, set main color of theme
%		to match Web site color:
\definecolor{MainColor}{RGB}{10, 74, 109}
\colorlet{MainColorMedium}{MainColor!50}
\colorlet{MainColorLight}{MainColor!20}
\usecolortheme[named=MainColor]{structure} 

%	For tables
%[dvipsnames] [table]
\usepackage{xcolor}
\usepackage{tabu}	% Even fancier than tabulary
\usepackage{multirow}

%	Just for the degree symbol
\usepackage{textcomp}

%	Get rid of footline (page, author, etc. on each slide)
\setbeamertemplate{footline}{}
%	Get rid of navigation buttons
\setbeamertemplate{navigation symbols}{}

%	Make footnotes not ugly
\usepackage{hanging}
\setbeamertemplate{footnote}{\raggedright\hangpara{1em}{1}\makebox[1em][l]{\insertfootnotemark}\footnotesize\insertfootnotetext\par}

%	Text style for code snippets inline in text:
\newcommand{\codeInline}[1]{\texttt{#1}}

%	Text style for emphasis stronger than \emph:
%		(Note, this doesn't toggle the way \emph does.
%			(Note, this can be done, didn't seem worth the trouble.))
\newcommand{\strong}[1]{{\bfseries{#1}}}


%	******	Define title page	**********************
\setbeamertemplate{title page}{
	{\color{MainColor}
	% There must be a better way than this -vspace at
	%	 the top and bottom of the page to reduce the 
	%	 bottom margin, but I can't find one that works.
	\vspace{-6em}

	% Go to a lot of trouble to get the title in a
	%	nice box, since customizing a beamer block
	%	does not entirely work here (I don't know why)
	\newlength{\titleBoxWidth}
	\setlength{\titleBoxWidth}{\textwidth}
	\addtolength{\titleBoxWidth}{-2.0em}
	\setlength{\fboxsep}{1.0em}
	\setlength{\fboxrule}{0pt}
	\fcolorbox{MainColor!25}{MainColor!25}{
		\parbox{\titleBoxWidth}{
			\raggedright
			\LARGE\textbf{\inserttitle}
		}	% end parbox
	}	% end fcolorbox

	\vfill
	\small{Author: \insertauthor}
	\vspace{\baselineskip}

	\small{\Course}

	\small{\Instructor}
	\vspace{\baselineskip}

	\small{\emph{This material is part of the \strong{statsTeachR} project}}

	\vspace{0.33\baselineskip}\scriptsize{\emph{\LicenseText}}





		\vspace{-15em}

	}	% end color
	\clearpage
}	% end define title page




%	The following variables are assumed by the standard preamble:
%	Global variable containing module name:
\title{Module Name}
%	Global variable containing module shortname:
%		(Currently unused, may be used in future.)
\newcommand{\ModuleShortname}{shortName}
%	Global variable containing author name:
\author{Gregory J. Matthews}
%	Global variable containing text of license terms:
\newcommand{\LicenseText}{Made available under the Creative Commons Attribution-ShareAlike 3.0 Unported License: http://creativecommons.org/licenses/by-sa/3.0/deed.en\textunderscore US }
%	Instructor: optional, can leave blank.
%		Recommended format: {Instructor: Jane Doe}
\newcommand{\Instructor}{}
%	Course: optional, can leave blank.
%		Recommended format: {Course: Biostatistics 101}
\newcommand{\Course}{}



%	******	Document body begins here	**********************

\begin{document}

%	Title page
\begin{frame}[plain]
	\titlepage
\end{frame}

%	******	Everything through the above line must be placed at
%		the top of any TeX file using the statsTeachR standard
%		beamer preamble. 

\begin{frame}{Overview}
\begin{itemize}
\item stuff
\end{itemize}
\end{frame}


\begin{frame}{Accessing the MGHPCC}
\begin{itemize}
\item Register for the MGHPCC (for UMass): https://www.umassrc.org/hpc/
\item You need: a login and a password
\item Logging in: ssh username@ghpcc06.umassrc.org
\end{itemize}
\end{frame}


\begin{frame}{Accessing the MGHPCC}
\begin{itemize}
\item Register for the MGHPCC (for UMass): https://www.umassrc.org/hpc/
\item You need: a login and a password
\item Logging in: ssh username@ghpcc06.umassrc.org
\item NOTE: You can only access the MGHPCC if you are on UMass's network (physically on campus or VPN)
\end{itemize}
\end{frame}


\begin{frame}{Software}
\begin{itemize}
\item Once logged in, you will be placed in your home directory
\item In order to use software, you'll need to load modules.  
\item List of available software: http://wiki.umassrc.org/wiki/index.php/Provided\_Software
\item We are interested in using R here.  
\item So we need "module load R/3.0.1"
\item We can also unload a module.  "module unload R"
\end{itemize}
\end{frame}


\begin{frame}{LSF common commands}
\begin{itemize}
\item bsub - submit a job
\item bkill - kill a job
\item bjobs - view status of jobs
\item bpeek - view output / error files
\item bhist - job history
\item bqueues - available queues
\end{itemize}
\end{frame}




\begin{frame}{Batch Mode}
\begin{itemize}
\item Rather than running R code interactively, we can also run R in BATCH mode.  
\item At the prompt, we submit a job by typing: R CMD BATCH filename.R
\end{itemize}
\end{frame}



\begin{frame}{Batch Mode}
\begin{itemize}
\item Once we have a .R file that we want to run we can submit it to the LSF job scheduler.  
\item What actually gets submited to the LSF scheduler is a shell script.  
\item We need to create a file that contains all of the code we would have run at the shell prompt.  
\end{itemize}
{\tt > module load R/3.0.1 }\\
{\tt > R CMD BATCH filename.R}\\
\end{frame}




\end{document}